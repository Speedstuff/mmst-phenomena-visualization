\section*{Summary}
\addcontentsline{toc}{section}{Summary}

Throughout the project, we learned a lot about managing a complex task as a group and are satisfied with the accomplished prototype. Some details could be improved for the final product like mentioned in the validation section, mainly the ON/OFF switch and the positioning of the exhaust arrow. Apart from that, the visualizations and interactions were clear for all users. That could be a result of the iterative design process we chose to follow. For that, we met twice a week and discussed the state of the design and the progress made since the last meeting. Possible changes and options were discussed and, in some cases, we implemented multiple solutions to compare them. This way for example, we concluded that corrugated arrows are a better visualization for the heat flow than colored, straight arrows. 

In the beginning, we had difficulties grasping what the goal of the project is. While we knew about the main objective of a furnace visualization, we did not understand who the stakeholders of our system were. As we learnt from our first customer interviews, the visualization should not be based on any quantitative parameters and only display simple relations with no underlying physical model. Therefore, our first idea was that the phenomena visualization is a tool for non-specialists to understand the basic principles. It was later clarified that the users of our system are furnace operators that know the plant and work with it daily. That confused us, as we did not understand the use-case of displaying this simplified information and began to work on a \ac{SCADA}-like system that would fit into such an industrial setting. After a lot of work of designing the first mockup, we realized that this is not the solution the customer was looking for and we needed to start from scratch. That whole process showed us, how crucial it is to clarify all specifications and requirements before starting to work on the solution itself. This way, a lot of time and work can be saved. 

Another important lesson we learnt was group management and task distribution. With increasing manpower it gets more challenging to keep everyone up-to-date and informed about who is working on certain subjects. An online tool we found most useful is called “Slite”\footnote{Slite Website: https://www.slite.com/}. In there, we created a protocol for every meeting to record where we currently stand in our development cycle and wrote down to-dos for each member of our group. This allowed us to keep track of everything end finish the project with no crunch time in the end. A nice side effect was, that ticking the checkboxes of the to-do list was very satisfying as everyone felt that progress was being made.

To write the report, we used “ShareLaTeX”\footnote{ShareLaTeX Website: https://tex.zih.tu-dresden.de/} which is a great tool hosted by the TU Dresden allowing us to work on our report simultaneously. We split up the document into six sections and distributed them evenly between us. After everyone had written their part, we combined them into the report and proofread it together.

The final prototype could be used in many applications. It visualizes the behavior of a furnace as required and can be understood by untrained people as well. Therefore, we think that its use-case extends beyond an industrial application, and it could be used for educational purposes, in apprenticeships, or at visitor centers of a plants. The modularity of the design would allow for further extension by adding more critical cases, like incomplete combustions that create toxic gases and reduce the efficiency, or applying an underlying physical model. There are many possibilities to adapt or extend the design and we hope that some ideas might find their way into real applications.
