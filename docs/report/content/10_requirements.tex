\section*{Requirements}
\addcontentsline{toc}{section}{Requirements}

 A highly reliable \ac{HMI} system which delivers safe, consistent, and intuitive performance relies on engineering best practices throughout design, panel layout, production, validation, and quality assurance \cite{DesignHMI}. As learned from the \ac{MMST}-lecture, clear definitions of the requirements, the operators level of expertise, and any interactions with other systems provide the starting point in a knowledge-intensive user-centered design process. 

This project is about designing a prototype of the system. The main benefit of prototyping is that it results in a faster and more effective design cycle. Because prototypes allow to test the design in the “real-world” environment, it is easier to identify potential problems and prevent costly mistakes down the road.

As mentioned in the introduction, the system worked on in this project, is an industrial furnace. To achieve a good design for the \ac{HMI} of the system, the prototype should have clarity, correctness and be intuitive. Moreover, complex processes happening in the furnace need to be illustrated, easily understandable and the controllability needs to be simple and intuitive as well.

Beside non-functional requirements, functional ones are essential and fundamental for the existence of the product, and express the main task, what the product has to do.
Designing an intuitive and easy controllable \ac{HMI} is very important. The design for this project needs to have an adjustable stream-, fuel- and airflow. In addition a power control, that switches the running state for the whole process, should be implemented.

The prototype being designed in this project should display the inputs and outputs. They need to have a logical connection to each other. For visualising the phenomena inside the furnace, these elements need to be shown:

\begin{itemize}
  \item Temperature of the feed and the output
  \item Size of the combustion reaction 
  \item Flow rate of the stream inside the pipe
  \item The amount of heat transfer, including heat loss and combustion products
\end{itemize}

Apart from the visualizations inside the furnace, the design should also display warnings of critical adjustments, indicating dangerous situations. To give the operator a good understanding of the feed stream state, the prototype should also include a logical link between the outlets temperature and volume.

To allow for a better understanding, relations between certain components need to be wisely connected. The most important relations for the design of the prototype are:

\begin{itemize}
  \item When the flow rate of the fuel to the furnace increases, and so does the released heat, the temperature of the outlet stream increment.
  \item Increasing the feed flow rate results in a decrease in the outlet stream temperature.
\end{itemize}
